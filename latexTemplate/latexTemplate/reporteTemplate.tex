%%
%% This is file `sample-sigconf.tex',
%% generated with the docstrip utility.
%%
%% The original source files were:
%%
%% samples.dtx  (with options: `sigconf')
%% 
%% IMPORTANT NOTICE:
%% 
%% For the copyright see the source file.
%% 
%% Any modified versions of this file must be renamed
%% with new filenames distinct from sample-sigconf.tex.
%% 
%% For distribution of the original source see the terms
%% for copying and modification in the file samples.dtx.
%% 
%% This generated file may be distributed as long as the
%% original source files, as listed above, are part of the
%% same distribution. (The sources need not necessarily be
%% in the same archive or directory.)
%%
%% The first command in your LaTeX source must be the \documentclass command.
\documentclass[sigconf]{acmart}
\usepackage[spanish]{babel}
\usepackage[utf8]{inputenc}
\usepackage{lmodern}
\usepackage{tabularx}
\usepackage{listings}




%%
%% \BibTeX command to typeset BibTeX logo in the docs
\AtBeginDocument{%
  \providecommand\BibTeX{{%
    \normalfont B\kern-0.5em{\scshape i\kern-0.25em b}\kern-0.8em\TeX}}}





%%
%% end of the preamble, start of the body of the document source.
\begin{document}

%%
%% The "title" command has an optional parameter,
%% allowing the author to define a "short title" to be used in page headers.
\title{Práctica 1}

%%
%% The "author" command and its associated commands are used to define
%% the authors and their affiliations.
%% Of note is the shared affiliation of the first two authors, and the
%% "authornote" and "authornotemark" commands
%% used to denote shared contribution to the research.
\author{Jorge Humberto Sierra Florido}
\email{2123065656@cua.uam.mx}
\affiliation{
  \institution{UAM Cuajimalpa \\ Ingeniería en Computación}
  \city{Ciudad de México}
  \country{México}
}

\author{María de Jesús Sánchez Zepeda}
\email{2153068423@cua.uam.mx}
\affiliation{
  \institution{UAM Cuajimalpa\\ Ingeniería en Computación}
  \city{Ciudad de México}
  \country{México}
}






%%
%% The abstract is a short summary of the work to be presented in the
%% article.
\begin{abstract}
  Aqu{\'i} va el abstract de la pr{\'a}ctica...
\end{abstract}




%%
%% This command processes the author and affiliation and title
%% information and builds the first part of the formatted document.
\maketitle

\section{Introducción}

En el análisis de datos es importante la limpieza de los datos previos a realizar modelos predictivos con ellos. Con limpieza nos referimos al tratamiento de los datos faltantes con el fin de no inducir un sesgo en el modelo.

Un método para encontrar relaciones entre varias variables de un conjunto de datos datos es la regresión líneal.

Se tienen dos conjuntos de datos de diferentes tamaños a los cuáles se les desea encontrar un modelo una ecuación que modele la relación entre variables determinadas.

Para ello se limpiarán los datos de cada conjunto, se realizará una regresión lineal para posteriormente evaluar dicho modelo.
Texto introductorio al tema en que se enfoca la pr{\'a}ctica y lo que se desarrollar{\'a} en ella. Se debe escribir un texto que introduzca el tema de la 
pr{\'a}ctica, definici{\'o}n del problema, los objetivos, motivaci{\'o}n,  y resultados esperados.

\section{Conceptos previos}

%Escribir conceptos te{\'o}ricos empleados en el desarrollo de la pr{\'a}ctica (f{\'o}rmulas matem{\'a}ticas, por ejemplo). Es un tipo de secci{\'o}n con todos los conceptos te{\'o}ricos empleados.

\section{Metodología}

Se inicio la practica revisando los datasets proporcionados, estos corresponden a los datos de temperatura y salinidad medidas, así como las características de algunos vehículos. 

El primer paso fue cargar los datos en R a fin de poder analizar los datasets. para esta carga se usaron las siguientes instrucciones:

\begin{lstlisting}[caption=lectura de los datos]
data = read.csv("water.csv", header=TRUE)
data2 = read.table("mtcars.txt", header=TRUE , sep = " ")
\end{lstlisting}

Una vez cargada la información se procedió a obtener el resumen de la misma con la siguiente instrucción(\ref{tab:Tabla1})

\begin{lstlisting}
summary(data)
\end{lstlisting}

\begin{table}
	\begin{tabularx}{\columnwidth}{|X|X|X|X|}
		\hline
		\multicolumn{2}{|c|}{ TdegC } & \multicolumn{2}{|c|}{ Salnty }  \\
		\hline
	Min. & 1.44  & Min. & 28.43 \\
		\hline
	1st Qu.& 7.68 &1st Qu.&33.49 \\
		\hline
	Median &10.06  & Median & 33.86   \\
		\hline
	Mean & 10.80 & Mean & 33.84\\
		\hline
	3rd Qu.&13.88 & 3rd Qu. & 34.20\\
		\hline
	Max. & 31.14 & Max.& 37.03  \\
		\hline
	NA's & 10963 & NA's  & 47354\\
		\hline
	\end{tabularx}
	\caption{\label{tab:Tabla1}Resumen de el dataset water.csv}
\end{table}

En la cual podemos observar que existen varios registros vacíos, 10963 en el campo de temperatura y 47354 en la salinidad

\section{Resultados}

Describir con detalle todos los resultados. Hacer una discusi{\'o}n de lo obtenido. Se debe mostrar un enfoque anal{\'i}tico sobre los resultados generados. Mostrar absolutamente todos los productos obtenidos debido a los pasos mostrados en la metodolog{\'i}a.

\section{Conclusiones y reflexiones}

Conclusiones generales de la pr{\'a}ctica. A\~nadir una reflexi{\'o}n anal{\'i}tica por cada miembro del equipo.

%%
%% The next two lines define the bibliography style to be used, and
%% the bibliography file.
\bibliographystyle{ACM-Reference-Format}
\bibliography{references}


\end{document}
\endinput
%%
%% End of file `sample-sigconf.tex'.
